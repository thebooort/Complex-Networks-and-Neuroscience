
\addcontentsline{toc}{chapter}{Abstract}

\begin{abstract}

Este trabajo compone la resolucion de algunos de los ejercicios propuestos en la asignatura del Master de Fisica y Matematicas fisymat de la Universidad de Granada denominada Fisica de redes complejas. La primera parte de la asignatura es impartida por el profesor Joaquín Torres. 

Generalmente la mayoría de los ejercicios poseen algún componente computacional, ya sean cálculos, simulación o visualización. Con el fin de resaltar la importancia de este aspecto, pero no dificultar la lectura, los códigos están adjuntos y se pueden consultar en los apéndices dejando solo para el grueso del texto las conclusiones derivadas y las visualizaciones. 

Así mismo muchos de los códigos empleados tambien poseen un salida visual en forma de animaciones y videos que puede ser interesante observar. Para este fin puede consultar el respositorio en github donde están todos los códigos utilizados, las salidas de los mismos y por supuesto, una versión de este documento en pdf y latex bajo la licencia creative commons:

\url{http://www.github.com/thebooort/Complex-Networks-and-Neuroscience}
 
 Algunos de los códigos utilizados utilizan partes de otras librerías o códigos. En esos casos quedan citados los otros responsables y la licencia se adhiere a la que pusieran sus creadores iniciales antes de mis modificaciones para este trabajo.
 

\end{abstract}
